% ---
% RESUMOS
% ---

% RESUMO em português
\setlength{\absparsep}{18pt} % ajusta o espaçamento dos parágrafos do resumo
\begin{resumo}
Nos últimos anos, dados conectados têm sido a forma mais proeminente para abertura de dados em diversos países. Tal forma utiliza padrões para descrição destes dados, promovendo a sua interoperabilidade, seu reuso e a sua integração. No entanto, integrar a informação entre diferentes conjuntos de dados surge como um empecilho para o seu desenvolvimento, principalmente se tal integração consistir na correspondência de uma determinada entidade do mundo real em conjuntos de dados distintos. Neste contexto, este trabalho propõe uma abordagem para auxiliar na identificação de instâncias correspondentes. Para isso, baseia-se na modelagem conceitual dos dados, permitindo que os relacionamentos entre os conceitos sejam utilizados para descobrir novas correspondências entre os dados. Para avaliar a eficácia da proposta foram realizados um estudo de caso e um experimento.  No estudo de caso, a proposta foi utilizada para encontrar as correspondências de pesquisadores e publicações em quatro \textit{datasets} (Lattes, RBIE, SBIE e WIE) e, então, responder um conjunto com mais de trinta perguntas realizadas pela comunidade de Informática na Educação. No experimento, a proposta foi utilizada em dois cenários (C1 e C2) e comparada a outras abordagens através das métricas de precisão, revocação e medida-f. De acordo com os resultados apresentados, a proposta posicionou-se em primeiro e segundo lugar nos cenários C1 e C2 respectivamente, mesmo não utilizando computações específicas para os \textit{datasets}, permitindo sua utilização em outros contextos com o mínimo de esforço.
% * <profsean@gmail.com> 2017-01-18T15:29:55.530Z:
% 
% > correspondência de instâncias
% ou alinhamento de dados conectados? Veja o título da dissertação...
% 
% ^.

 \textbf{Palavras-chaves}: Correspondência de Instâncias, Alinhamento de Dados, \textit{Datasets}, Dados Conectados, Web de Dados.
\end{resumo}

% ABSTRACT in english
%\begin{resumo}[Abstract]
% \begin{otherlanguage*}{english}
%   TODO.
%
%   \vspace{\onelineskip}
% 
%   \noindent 
%   \textbf{Keywords}: TODO.
% \end{otherlanguage*}
%\end{resumo}