% ---
% RESUMOS
% ---

% RESUMO em português
\setlength{\absparsep}{18pt} % ajusta o espaçamento dos parágrafos do resumo
\begin{resumo}
Atualmente várias iniciativas de abertura de dados vêm sendo realizadas ao redor do mundo, dessas iniciativas a maior parte é provida pelo segmento governamental através da publicação de seus dados. permitindo que governo e sociedade civil faça uso das informações das mais diversas formas. Por Exemplo, no desenvolvimento de aplicações para melhorar a tomada de decisão por parte do governo, bem como, em aplicações que possam ser utilizadas diretamente pela sociedade. 

Apesar dos benefícios providos pela crescente oferta de dados, é possível ressaltar problemas que ainda estão em aberto, como, por exemplo, o baixo reuso dos dados publicados, baixa integração e não utilização de padrões para a descrição desses dados, acarretando em uma baixa interoperabilidade. Baseando-se nessas questões, abordagens que objetivem integrar dados ou que permitiam uma maior interoperabilidade  merecem destaque. Dessa forma, Dados Conectados se torna uma alternativa considerável para a resolução de tais questões.

Um problema existente na comunidade de Dados Conectados relacionado à integração de informação entre diferentes \textit{datasets} é chamado de problema de correspondência de instâncias do inglês (\textit{Instance Matching}), que se trata de encontrar instâncias em \textit{datasets} diferentes que se referem a mesma entidade do mundo real. Neste contexto, este trabalho propõe uma abordagem para a correspondência de instâncias.

 \textbf{Palavras-chaves}: Correspondência de Instâncias, Alinhamento de Dados, \textit{Datasets}, Dados Conectados, Web de Dados.
\end{resumo}

% ABSTRACT in english
\begin{resumo}[Abstract]
 \begin{otherlanguage*}{english}
   TODO.

   \vspace{\onelineskip}
 
   \noindent 
   \textbf{Keywords}: TODO.
 \end{otherlanguage*}
\end{resumo}