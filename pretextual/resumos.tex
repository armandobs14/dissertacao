% ---
% RESUMOS
% ---

% RESUMO em português
\setlength{\absparsep}{18pt} % ajusta o espaçamento dos parágrafos do resumo
\begin{resumo}
Atualmente várias iniciativas de abertura de dados vêm sendo realizadas ao redor do mundo, das quais a maior parte é provida pelo segmento governamental. Uma das formas de se promover a abertura de dados é através de Dados Conectados, que a partir da utilização de padrões para a descrição dos dados visa apoiar a interoperabilidade, reuso e integração dos dados. Entretanto, um problema existente na comunidade de Dados Conectados relacionado à integração de informação entre diferentes conjuntos de dados (\textit{datasets}), consiste em encontrar correspondências de instâncias do inglês (\textit{Instance Matching}), ou seja, encontrar instâncias em \textit{datasets} diferentes que se referem a mesma entidade do mundo real, fazendo os respectivos alinhamentos dessas instâncias. Neste contexto, este trabalho propõe uma abordagem para auxiliar na identificação de instâncias correspondentes. Para isso, baseia-se na modelagem conceitual dos dados, permitindo que os relacionamentos entre os conceitos sejam utilizados para descobrir novas correspondências entre os dados. Para avaliar a eficácia da proposta foram realizados um estudo de caso e um experimento.  No estudo de caso, a proposta foi utilizada para encontrar as correspondências de pesquisadores e publicações em quatro \textit{datasets} (Lattes, RBIE, SBIE e WIE) e, então, responder um conjunto com mais de trinta perguntas realizadas pela comunidade de Informática na Educação. No experimento, a proposta foi utilizada em dois cenários (C1 e C2) e comparada a outras abordagens através das métricas de precisão, revocação e medida-f. De acordo com os resultados apresentados, a proposta posicionou-se em primeiro e segundo lugar nos cenários C1 e C2 respectivamente, mesmo não utilizando computações específicas para os \textit{datasets}, permitindo sua utilização em outros contextos com o mínimo de esforço.


 \textbf{Palavras-chaves}: Correspondência de Instâncias, Alinhamento de Dados, \textit{Datasets}, Dados Conectados, Web de Dados.
\end{resumo}

% ABSTRACT in english
%\begin{resumo}[Abstract]
% \begin{otherlanguage*}{english}
%   TODO.
%
%   \vspace{\onelineskip}
% 
%   \noindent 
%   \textbf{Keywords}: TODO.
% \end{otherlanguage*}
%\end{resumo}