% ---
% RESUMOS
% ---

% RESUMO em português
\setlength{\absparsep}{18pt} % ajusta o espaçamento dos parágrafos do resumo
\begin{resumo}
Atualmente várias iniciativas de abertura de dados vêm sendo realizadas ao redor do mundo, dessas iniciativas a maior parte é provida pelo segmento governamental através da abertura de dados. Apesar dos benefícios providos pela crescente oferta de dados, é possível ressaltar problemas que ainda estão em aberto, como, por exemplo, o baixo reuso dos dados publicados, baixa integração e não utilização de padrões para a descrição desses dados, acarretando em uma baixa interoperabilidade. Baseando-se nessas questões, abordagens que objetivem integrar dados ou que permitiam uma maior interoperabilidade  merecem destaque. Dessa forma, Dados Conectados se torna uma alternativa considerável para a resolução de tais questões.
Um problema existente na comunidade de Dados Conectados relacionado à integração de informação entre diferentes \textit{datasets} é chamado de problema de correspondência de instâncias do inglês (\textit{Instance Matching}), que se trata de encontrar instâncias em \textit{datasets} diferentes que se referem a mesma entidade do mundo real. Neste contexto, este trabalho propõe uma abordagem para a correspondência de instâncias.
A abordagem proposta visa auxiliar na identificação de instâncias correspondentes. Para isso, a proposta baseia-se na modelagem conceitual utilizada nos dados, permitindo que o relacionamento entre os conceitos sejam utilizados para descobrir novas correspondências entre os dados. Para avaliar a eficácia da proposta foram realizados um estudo de caso e um experimento.  No estudo de caso, a proposta foi utilizada para encontrar a correspondência de pesquisadores e publicações em quatro \textit{datasets} (Lattes, RBIE, SBIE e WIE). A partir das correspondências geradas foram utilizadas para responder um conjunto com mais de trinta perguntas realizadas pela comunidade de Informática na Educação.
No experimento, a proposta foi utilizada em dois cenários. No cenário C1 os dados apresentam cerca de 9 heterogeneidades, sendo elas de multilinguagem, diferenças entre catálogos, diferença no grau de descrição e outros. No cenário C2 os dados apresentam grupos de instâncias similares, no qual existe apenas uma correspondência correta. A comparação entre as abordagens foi realizada diante da eficácia  que é composta pelas métricas de precisão, revocação e medida-f. 
De acordo com os resultados apresentados, a proposta posicionou-se em primeiro e segundo lugar nos cenários C1 e C2 respectivamente. Apesar de não se destacar em ambos os cenários, vale ressaltar que a proposta apresentada trata-se de uma abordagem que não utiliza computações específicas para o \textit{dataset}, permitindo sua utilização em outros contextos om o mínimo de esforço.

 \textbf{Palavras-chaves}: Correspondência de Instâncias, Alinhamento de Dados, \textit{Datasets}, Dados Conectados, Web de Dados.
\end{resumo}

% ABSTRACT in english
%\begin{resumo}[Abstract]
% \begin{otherlanguage*}{english}
%   TODO.
%
%   \vspace{\onelineskip}
% 
%   \noindent 
%   \textbf{Keywords}: TODO.
% \end{otherlanguage*}
%\end{resumo}