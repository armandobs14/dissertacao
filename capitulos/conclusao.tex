\chapter{Considerações finais e trabalhos futuros}
\label{cap:conclusao}
O trabalho apresentou uma abordagem semiautomática para alinhar \textit{datasets} reais. Através de diagramas de atividades e de componentes foi descrito o processo utilizado para conectar recursos de diferentes \textit{datasets}. Essa proposta se faz necessária justamente pela necessidade de soluções capazes de alinhar dados de forma confiável e com menor conhecimento possível do domínio. Além disso, a solução permite que o alinhamento seja executado diretamente dentro do armazenamento de triplas, não havendo a necessidade de gerar arquivos para alinhar. 

Visando avaliar a abordagem proposta em um cenário real, um estudo de caso foi desenvolvido. Neste estudo, a proposta foi utilizada para alinhar \textit{datasets} do RBIE, SBIE, WIE e Lattes. A utilização da solução de alinhamento permitiu que diversas perguntas pudessem ser respondidas. Além disso, foi possível problemas de erros e insistências nos dados puderam ser encontrados.

Por fim, um experimento foi conduzido  a fim de avaliar e comparar em termos de eficácia através das métricas de precisão, revocação e medida-f. Essas métricas foram avaliadas em dois cenários de alinhamento. O experimento apresentou resultados preliminares satisfatórios com base no desempenho dos trabalhos relacionados, pois em todos os cenários a proposta foi capaz de identificar um número significativo de trabalhos que poderiam ser alinhados. 

\section{Principais contribuições}
As principais contribuições deste trabalho são apresentadas a seguir:
\begin{itemize}
\item Desenvolvimento de processo alinhamento de dados conectados, permitindo que \textit{datasets} reais pudessem ser conectados;
\item Abordagem capaz de calcular a similaridade entre recursos levando em consideração o modelo ontológico;
\item Viabilização da execução do alinhamento diretamente dentro do armazenamento de triplas.
\item  Criação de experimento e estudo de caso para avaliar a eficácia das soluções de alinhamento no estado da arte.
\end{itemize}

\subsection{Limitações e trabalhos futuros}

Algumas questões que não foram foco deste trabalho, mas que devem ser consideradas em trabalhos futuros, são estudos em como a qualidade e o tamanho (quantidade de triplas) do \textit{dataset} impacta na qualidade dos alinhamentos.

Outra questão que não está dentro do escopo deste trabalho, mas que também deve ser levado em consideração, é o alinhamento entre ontologias, pois este trabalho foi desenvolvido com objetivo de alinhar dados. Por tanto, se faz necessário estudar mecanismos que unifiquem soluções para alinhar dados e alinhar ontologias.

Como trabalhos futuros, pretende-se realizar mais experimentos para analisar a eficácia da ferramenta com \textit{datasets} com diversas características (Domínio, quantidade de triplas, qualidade etc.). Além disso, pretende-se disponibilizar a solução como uma infraestrutura de alinhamento disponível na Web, dando acesso aos alinhamentos realizados através da solução. Por fim, outras pesquisas serão aplicadas buscando automatizar ainda mais o processo de alinhamento, mais especificamente a escolha dos conceitos buscando otimizar o desempenho e a qualidade nos alinhamentos obtidos.
