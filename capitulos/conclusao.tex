\chapter{Considerações finais e trabalhos futuros}
\label{cap:conclusao}
O trabalho apresentou uma abordagem semiautomática para alinhar \textit{datasets} reais. Através de diagramas de atividades e de componentes foi descrito o processo utilizado para conectar recursos de diferentes \textit{datasets}. Essa proposta se faz necessária justamente pela necessidade de soluções capazes de alinhar dados de forma confiável e com menor conhecimento possível do domínio. Além disso, a solução permite que o alinhamento seja executado diretamente dentro do armazenamento de triplas, não havendo a necessidade de gerar arquivos para alinhar. 

Visando avaliar a abordagem proposta em um cenário real, um estudo de caso foi desenvolvido. Neste estudo, a proposta foi utilizada para alinhar \textit{datasets} do RBIE, SBIE, WIE e Lattes. A utilização da solução de alinhamento permitiu que diversas perguntas pudessem ser respondidas. Além disso, foi possível notar problemas relacionados às informações que eram fornecidas por autores que submetiam seus trabalhos.

Por fim, um experimento foi conduzido a fim de avaliar a proposta e compara-la em termos de eficácia com outras ferramentas através das métricas de precisão, revocação e medida-f. Essas métricas foram avaliadas em dois cenários de alinhamento, nos quais a proposta obteve primeiro e segundo lugar. Apesar de não ter tido os melhores valores em ambas as avaliações, a proposta apresentada se destaca pela ausência de implementações específicas para o \textit{dataset}, gerando menos impacto quando é necessária uma mudança de contexto.

\section{Principais contribuições}
As principais contribuições deste trabalho são apresentadas a seguir:
\begin{itemize}
\item Desenvolvimento de processo alinhamento de dados conectados, permitindo que \textit{datasets} reais possam ser conectados semi-automaticamente;
\item Abordagem capaz de calcular a similaridade de recursos levando em consideração o modelo ontológico;
\item Viabilização da execução do alinhamento diretamente dentro do armazenamento de triplas;
\item  Criação de experimento e estudo de caso para avaliar a eficácia das soluções de alinhamento no estado da arte.
\end{itemize}

Em linhas gerais, tem-se que esta proposta pode ser aplicada a diferentes domínios, permitindo que seja possível extrair contribuições em cada um deles:

\begin{itemize}
\item \textbf{Contribuições para Informática na Educação (estudo de caso):} foi possível gerar um panorama da comunidade, permitindo que através do cruzamento de dados, a comunidade tenha uma visão mais geral e através das respostas fornecidas será possível adotar decisões estratégicas.
\item \textbf{Dados Conectados:} A partir da utilização do processo, é possível construir uma abordagem livre de contexto para a correspondência de instâncias. Essa abordagem pode ser aplicada em processos de publicação de dados conectados, permitindo que os dados sejam enriquecidos durante a publicação. Além disso, o consumo de dados conectados pode explorar o processo, visto que os dados podem ser alinhados com possíveis \textit{datasets} locais.
\item \textbf{Governo:} A abordagem permite que os dados sejam enriquecidos através da correspondência entre instâncias, contribuindo com iniciativas de interoperabilidade de dados.
\item \textbf{Cidadão} Através das correspondências geradas, a sociedade poderia identificar e monitorar com maior facilidade entidades públicas de seu interesse.
\end{itemize}
\subsection{Limitações e trabalhos futuros}

Algumas questões que não foram foco deste trabalho, mas que devem ser consideradas em trabalhos futuros, são estudos relacionados à qualidade e o tamanho (quantidade de triplas) do \textit{dataset} impacta na qualidade dos alinhamentos.

Outra questão que não está dentro do escopo deste trabalho, mas que também deve ser levado em consideração, é o alinhamento entre ontologias, pois este trabalho foi desenvolvido com objetivo de alinhar dados. Portanto, faz-se necessário estudar mecanismos que unifiquem soluções para alinhar dados e alinhar ontologias.

Com trabalhos futuros, pretende-se realizar mais experimentos para analisar a eficácia da ferramenta com \textit{datasets} com diversas características (domínio, quantidade de triplas, qualidade etc.). Além disso, pretende-se disponibilizar a solução como uma infraestrutura de alinhamento disponível na Web, dando acesso aos alinhamentos realizados através da solução. Por fim, outras pesquisas serão aplicadas com os seguintes objetivos:
\begin{itemize}
\item \textbf{Automatizar o processo de alinhamento:} Um possível caminho para isso seria a escolha automática dos conceitos relacionados;
\item \textbf{Otimizar a performance:} Como os conceitos relacionados podem ser alinhados paralelamente, possíveis abordagens seriam paralelismos e distribuição.
\item \textbf{Melhorar a qualidade no cálculo de similaridade entre recursos:} Uma abordagem possível seria a composição de funções de similaridade e a identificação de características que apresentem maior expressividade na identificação de similaridade.
\end{itemize}

