\chapter{Introdução}
\label{cap:introducao}
O Atual paradigma da Web é descrito como Web de documentos. Neste, os documentos ou páginas são descritas através de uma linguagem de marcação de hipertexto (HTML) e interligados através de hiperlinks. Para identificar as páginas na Web, é necessário um mecanismo capaz de identificar de forma única cada uma dessas páginas. Para isso, a identificação é realizada através de um mecanismo de identificação global chamado URI (Uniform Resource Identifier). Dessa forma, para ter acesso a um conteúdo específico de qualquer documento neste paradigma, é necessário obter todo o documento.

Diferentemente da Web de documentos, a Web de dados, através de URIs, possibilita que os dados sejam identificados de forma única, permitindo que apenas os dados desejados sejam recuperados. Além disso, os dados são representados através de um framework de descrição de recursos (RDF) e uma linguagem de consulta aos dados, linguagem SPARQL. Através desses recursos, é possível apontar e navegar entre os dados na Web. Surgindo o conceito de Dados Conectados.

%Deve-se ressaltar que o RDF é responsável por dar estrutura aos dados seguindo o padrão de tripla: Sujeito, Predicado e Objeto. Com objetivo de dar significado aos dados utilizam-se vocabulários e ontologias. Tecnicamente, pode-se entender uma ontologia como um artefato computacional utilizado para modelar um domínio de interesse. Essa ontologia é composta de classes e propriedades, que são implementados através de uma linguagem de descrição de ontologias (OWL).
 
O conceito de Dados Conectados está intimamente relacionado à Web de dados, pois de acordo com \citeonline{bizer2009linked}, Dados Conectados trata-se de usar a Web para criar links entre dados de fontes diferentes. Além disso, Dados Conectados é considerado um ponto chave para o desenvolvimento da Web Semântica \cite{berners2001semantic} . Além disso, \citeonline{Isotani2015} destacam seu potencial de aplicação, tando para negócios quanto para o governo.  

Caso como o da BestBuy, que través da utilização de serialização RDF (RDFa) melhorou o número de acessos via buscador  Google entre 15\% e 30\%. Além do próprio Google, que passou a utilizar a serialização json-ld em um de seus produtos, o Gmail. No governo, temos o caso do governo britânico que publica seus dados em formato RDF, permitindo que os dados possam ser conectados a outros, mesmo estando em bases diferentes. Consequentemente, a utilização de Dados Conectados vem sendo realizada de forma crescente.

Dados Conectados pode ser visto sob duas perspectivas: consumo e publicação. A primeira aborda o ponto de vista do consumidor, tratando a exploração de dados para tornar as aplicações mais ricas. A segunda está sob a perspectiva do publicador, abordando processos \cite{bizer2007publish, hyland2011joy, villazon2011methodological, Avila2015} e conceitos \cite{berners2006linked, wood2014linked} necessários para publicar e manter os dados na Web de forma conectada. 

Publicar ou manter dados conectados na Web vai além de disponibilizar conjuntos de dados através de serializações RDF. É necessário conectá-los a outros conjuntos de dados já existentes. Porém, criar links entre conjuntos de dados requer uma análise cuidadosa por parte do especialista, que apesar de ser uma abordagem eficaz, não é escalável, visto que a quantidade de dados publicados cresce constantemente. Consequentemente, inviabilizando o processo de publicação de forma artesanal. Logo, para que seja possível construir a Web de Dados de forma eficiente, é necessário que existam soluções capazes de conectar dados de forma automática ou semiautomática.

Conectar dados automaticamente é um problema reconhecido por diversas comunidades. Dentre essas comunidades, podemos destacar as comunidades de Bancos de Dados e Web Semântica. Na primeira, esse problema é conhecido através do termo \textbf{\textit{Record Linkage}} \cite{gu2003record}. Na segunda, ele é reconhecido pelo termo \textbf{\textit{Instance Matching}}. Além disso, também é possível encontrar referências como \textbf{\textit{Problema de Resolução de Entidades}} \cite{menestrina2005generic} e \textbf{\textit{Deduplicação}} \cite{sarawagi2002interactive}, que se trata de um processo que tem como objetivo identificar e conectar recursos julgados representar a mesma entidade do mundo real.

\citeonline{castano2011ontology} ressaltam que a correspondência de instâncias (Instance Matching) apresenta características adicionais tais como: (i) \textbf{heterogeneidade estrutural}, que se refere à variação na estrutura das instâncias; (ii)\textbf{ conhecimento implícito}, que se refere as características e restrições  apresentadas pelo domínio e (iii) \textbf{identificação orientada a URI}, refere-se ao reuso das URIs para identificar novas informações a respeito de instâncias já existentes. Logo, há a necessidade de soluções específicas para a execução correta do processo de correspondência de instâncias. 

Para identificar e conectar recursos na Web, a comunidade vem apresentando um número crescente de soluções (ver Figura \ref{fig:oaei_imtools}).a \textit{Ontology Alignment Evaluation Initiative} (OAEI) realiza uma avaliação anual, que consiste em alinhar dois conjuntos de dados pré-definidos e comparar o alinhamento gerado pela solução com o alinhamento de referência. A partir da comparação entre os dois conjuntos de alinhamento, as métricas de precisão, revocação e medida-f são geradas.

\begin{figure}[!h]
        \centering
        \includegraphics[width=0.9\textwidth]{./imagens/im_tools.pdf}
    \caption{Quantidade de soluções submetidas ao OAEI}
        \footnotesize{Fonte: \cite{cheatham2015results}.}
        \label{fig:oaei_imtools}
\end{figure}

Porém, de acordo com \citeonline{homoceanu2014putting}, apesar dos bons resultados apresentados, as soluções não estão prontas para alinhar dados automaticamente de forma confiável. Para isso, foi realizado um experimento equivalente ao executado pela OAEI. Entretanto, utilizando dados reais, que foram obtidos de 5 fontes diferentes (Freebase\footnote{\url{http://www.freebase.com/}}, DBPedia\footnote{\url{http://dbpedia.org}} , LinkedMDB\footnote{\url{http://www.linkedmdb.org}}, DrugBase\footnote{\url{http://www.drugbase.de/de/}} e NewYork Times\footnote{\url{http://www.nytimes.com}}). Além disso, o experimento permitiu notar que as características do modelo não eram considerados durante o processo de correspondência de instâncias.

Segundo \cite{homoceanu2014putting} e \cite{ferrara2008towards} uma solução de correspondência de instâncias deveria explorar a modelagem ontológica, que utiliza vocabulários e ontologias para seu desenvolvimento. Essa recomendação se deve ao fato de RDF apenas dar estrutura aos dado. Sendo responsabilidades das ontologias dar significado. Um modelo ontológico é composto por classes e propriedades, que são descritas através de uma linguagem de descrição de ontologias (e.g. OWL). As classes são utilizadas para representar os conceitos que pertencem ao domínio de interesse, já as propriedades são utilizadas para relacionar conceitos, sendo chamadas de propriedades de objeto (object properties) ou relacionar conceitos e dados, sendo chamadas de data properties. Essas propriedades possuem características como transitividade, simetria e reflexibilidade. Dessa forma, o baixo suporte às características das propriedades pode afetar diretamente na qualidade de soluções de correspondência de instâncias. 
        
Dentre as propriedades com suporte inadequado,  destaca-se a propriedade owl:sameAs, que é responsável por identificar recursos equivalentes. Além disso, essa se trata de uma propriedade transitiva. Dessa forma, se existem dois recursos equivalentes R1 e R2 e existe um terceiro recurso R3 que é equivalente a R2, então R1 é equivalente a R3. Como representado na Figura \ref{sameAs}.Tais características fazem com que a propriedade owl:sameAs seja uma das  mais utilizadas para alinhar dados na Web. Dessa forma, utilizar ferramentas que levem em consideração as características das propriedades é de grande importância para alinhar dados de forma confiável.


\begin{figure}[h]
	\centering
	\subfloat[Before]
	{
		\begin{tikzpicture}[node distance=1cm, auto,]
		%nodes
		\node[ellipse,draw] (r3) {R3};
		
		\node[above=of r3] (dummy) {};
		\node[right= of dummy,ellipse,draw](r2) {R2}
		edge[pil,<->, bend left=45] node[auto] {owl:sameAs} (r3);
		
		\node[left= of dummy,ellipse,draw] (r1) {R1}
		edge[dashed,<->, bend right=45] node[auto] {owl:sameAs} (r3)
		edge[pil,<->, bend left=45] node[auto] {owl:sameAs} (r2);
		\end{tikzpicture}
	}
	\subfloat[After]
	{
		\begin{tikzpicture}[node distance=1cm, auto,]
		%nodes
		\node[ellipse,draw] (r3) {R3};
		
		\node[above=of r3] (dummy) {};
		\node[right= of dummy,ellipse,draw](r2) {R2}
		edge[pil,<->, bend left=45] node[auto] {owl:sameAs} (r3);
		
		\node[left= of dummy,ellipse,draw] (r1) {R1}
		edge[dashed,<->, bend right=45] node[auto] {owl:sameAs} (r3)
		edge[pil,<->, bend left=45] node[auto] {owl:sameAs} (r2);
		\end{tikzpicture}
		}
		\caption{Transitividade da propriedade owl:sameAs}
		\label{sameAsSample}
\end{figure}


Neste contexto, este trabalho propõe uma abordagem independente de contexto para o alinhamento de dados conectados por meio de um processo de alinhamento que leva em consideração aspectos dos dados e  características do modelo ontológico. Assim, os recursos/instâncias analisados, além de alinhados através das propriedades de dados, podem ser alinhados través seus relacionamentos. Ademais, a proposta trata o problema do alinhamento entre datasets reais, permitindo que seja possível alinhar datasets distribuídos na Web de forma confiável.

\section{Objetivo}

Essa abordagem visa disponibilizar um mecanismo útil que permita alinhar semiautomaticamente recursos entre datasets diferentes. Além disso, a proposta também pretende facilitar a identificação e alinhamento de recursos duplicados dentro do mesmo dataset.
Dessa forma, o trabalho lida com aspectos mais gerais, como facilitar o alinhamento de dados, quanto com problemas específicos, como a definição de métricas para a análise de similaridade entre recursos, que sejam capazes de suportar os problemas provenientes dos datasets reais. Apesar de ser um trabalho com enfoque em engenharia de software, as suas contribuições estão mais voltadas para a área de Dados Conectados. Segue algumas dessas contribuições:

\begin{itemize}
        \item Construção de um processo para alinhar dados conectados independente de contexto;
        \item Definição de uma função de similaridade que contempla problemas provenientes de datasets reais (acentuação, ausência de propriedades, formatação e outros);
\end{itemize}

\section{Estrutura do trabalho}

Esta dissertação está dividida em 7 capítulos. O Capítulo 1 introduz a problemática e os objetivos do trabalho proposto, enaltecendo a necessidade de uma abordagem capaz de alinhar dados conectados. No Capítulo 2, são apresentados os conceitos relacionados ao tema deste trabalho, como RDF, Ontologias, Dados Conectados, Algoritmos de similaridade e Alinhamento de Dados Conectados.

No Capítulo 3, são apresentados os trabalhos relacionados à abordagem proposta. Algumas abordagens de alinhamento são descritas bem como comparadas com a solução proposta. Na conclusão do capítulo é apresentada uma tabela comparativa entre a abordagem proposta e os trabalhos relacionados apresentados.

O Capítulo 4 mostra como a proposta foi desenvolvida, tanto no processo de alinhamento por intermédio de um diagrama de atividades como na arquitetura através de um diagrama de componentes. Neste capítulo, todas as etapas do processo bem como os componentes da arquitetura são descritos detalhadamente.

No Capítulo 5 é apresentado um estudo de caso. Nele é descrita a plataforma QIE, um sistema que cruza dados da Revista Brasileira de Informática na Educação (RBIE), Simpósio Brasileiro de Informática na Educação (SBIE) e Workshop de informática na Escola (WIE) com dados extraídos da plataforma LATTES. Além disso, é descrito como o processo foi utilizado para alinhar os dados dos pesquisadores e de suas produções científicas entre essas bases.

No Capítulo 6, um experimento foi projetado para avaliar, em termos de eficácia através das métricas de precisão, revocação e medida-f, a abordagem proposta, em comparação com RiMOM \cite{zhang2015rimom} , Lily \cite{wang2015lily} e LogMap \cite{jimenez2015logmap}. Cada conjunto de alinhamento gerado é avaliado e uma discussão geral é apresentada ao final do capítulo.

Por fim, no Capítulo 7, são apresentadas as considerações finais deste trabalho. bem como são definidos alguns trabalhos futuros.
