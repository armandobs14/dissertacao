\chapter{Introdução}
\label{cap:introducao}
Considerado como ponto chave para a Web Semântica \cite{berners2001semantic}, Dados Conectados vem sendo abordado de forma crescente por acadêmicos, governos e empresas. Tal crescimento se deve ao fato de Dados Conectados ser uma alternativa relevante para conectar bases de dados distribuídas na Web. Segundo \citeonline{hyland2011joy}, Dados Conectados se trata de um conjunto de boas práticas para publicar e conectar dados estruturados na Web. Ademais, \citeonline{berners2006linked} afirma que não se trata apenas de por dados na Web, mas fazer conexões entre eles, permitindo assim, que pessoas ou máquinas possam explorar a Web dos dados.

Web de Dados bem como Web dos Documentos, são termos utilizados para descrever o paradigma da Web atualmente, onde o primeiro, que tem como base o segundo, reflete a possibilidade de conectar os dados através de links. Sendo esta, uma possibilidade inexistente quando se fala de Web dos Documentos, pois nesta, os documentos apontam para outros documentos através de links.

Dados Conectados pode ser visto sob duas perspectivas: consumo e publicação. A primeira aborda o ponto de vista do consumidor, tratando a exploração de dados para tornar as aplicações mais ricas. A segunda está sob a perspectiva do publicador, abordando processos \cite{bizer2007publish, hyland2011joy, villazon2011methodological, Avila2015} e conceitos \cite{berners2006linked, wood2014linked} necessários para publicar e manter os dados na Web de forma conectada. 

Publicar ou manter dados conectados na Web vai além de disponibilizar conjuntos de dados através de serializações RDF. É necessário conectá-los a outros conjuntos de dados já existentes. Porém, criar links entre conjuntos de dados requer uma análise cuidadosa por parte do especialista, que apesar de ser uma abordagem eficaz, não é escalável, visto que a quantidade de dados publicados cresce constantemente. Consequentemente, inviabilizando o processo de publicação de forma artesanal. Logo, para que seja possível construir a Web de Dados de forma eficiente, é necessário que existam soluções capazes de conectar dados de forma automática ou semiautomática.

Conectar dados automaticamente é um problema reconhecido por diversas comunidades. Dentre essas comunidades, podemos destacar as comunidades de Bancos de Dados e Web Semântica. Na primeira, esse problema é conhecido através do termo \textbf{\textit{Record Linkage}} \cite{gu2003record}. Na segunda, ele é reconhecido pelo termo \textbf{\textit{Instance Matching}}. Além disso, também é possível encontrar referências como \textbf{\textit{Problema de Resolução de Entidades}} \cite{menestrina2005generic} e \textbf{\textit{Deduplicação}} \cite{sarawagi2002interactive}, que se trata de um processo que tem como objetivo identificar e conectar recursos julgados representar a mesma entidade do mundo real.

Para identificar e conectar recursos na Web, a comunidade vem apresentando um número crescente de soluções (ver Figura \ref{fig:oaei_imtools}). Com o objetivo de avaliá-las, a \textit{Ontology Alignment Evaluation Initiative} (OAEI) realiza uma avaliação anual, que consiste em alinhar dois conjuntos de dados pré-definidos e comparar o alinhamento gerado pela solução com o alinhamento de referência. A partir da comparação entre os dois conjuntos de alinhamento as métricas de precision, recall e f-measure são geradas.

\begin{figure}[!ht]
	\centering
	\includegraphics[width=1\textwidth]{./imagens/im_tools.pdf}
    \caption{Quantidade de soluções submetidas ao OAEI}
	\footnotesize{Fonte: \cite{cheatham2015results}.}
	\label{fig:oaei_imtools}
\end{figure}

Porém, de acordo com \citeonline{homoceanu2014putting}, apesar dos bons resultados apresentados, as soluções não estão prontas para alinhar dados automaticamente de forma confiável. Para isso, foi realizado um experimento equivalente ao executado pela OAEI. Entretanto, utilizando dados reais, que foram obtidos de 5 fontes diferentes (Freebase, DBPedia, LinkedMDB, DrungBase e NewYork Times). Além disso, o experimento permitiu notar que não era provido o suporte adequado a algumas propriedades dos vocabulários utilizados.
	
Dentre as propriedades com suporte inadequado, destaca-se a propriedade owl:sameAs, que é responsável por identificar recursos equivalentes. Além disso, essa se trata de uma propriedade transitiva. Dessa forma, se existem dois recursos equivalentes R1 e R2 e existe um terceiro recurso R3 que é equivalente a R2, então R1 é equivalente a R3. Como representado na Figura \ref{sameAs}.Tais características fazem com que a propriedade owl:sameAs seja uma das  mais utilizadas para alinhar dados na Web. Dessa forma, utilizar ferramentas que levem em consideração as características das propriedades é de grande importância para alinhar dados de forma confiável.

\begin{figure}[h]
\centering
\begin{tikzpicture}[node distance=1cm, auto,]
 %nodes
 \node[ellipse,draw] (r3) {R3};
 
 \node[above=of r3] (dummy) {};
 \node[right= of dummy,ellipse,draw](r2) {R2}
 	edge[pil,<->, bend left=45] node[auto] {owl:sameAs} (r3);
 
 \node[left= of dummy,ellipse,draw] (r1) {R1}
  	edge[dashed,<->, bend right=45] node[auto] {owl:sameAs} (r3)
   	edge[pil,<->, bend left=45] node[auto] {owl:sameAs} (r2);
\end{tikzpicture}
\caption{Transitividade da propriedade owl:sameAs}
\label{sameAs}
\end{figure}

Neste trabalho, nós apresentamos uma abordagem semiautomática para  alinhamentar dados conectados baseada em modelagem semântica e algoritmos de similaridade. A partir dessas funções foi possível calcular a semelhança entre recursos/instâncias levando em consideração sua modelagem ontológica. Dessa forma, pretende-se permitir que datasets reais possam ser alinhados de forma confiável.

\subsection*{Contextualização}

\subsection*{Problemática}

\subsection*{Solucionática}

\subsection*{Objetivos}

\subsubsection*{Objetivo Geral}

\begin{itemize}
	\item Objetivo geral.
\end{itemize}

\subsubsection*{Objetivos específicos}
\begin{itemize}
\item Objetivo específico 1;
\item Objetivo específico 2;
\item Objetivo específico 3;
\item Objetivo específico 4.
\end{itemize}

\subsection*{Estrutura do trabalho}

O restante do trabalho está estruturado como segue:

\begin{enumerate}
\item[a)] \textbf{Seção 2} - descrição;
\item[b)] \textbf{Seção 3} - descrição;
\item[c)] \textbf{Seção 4} - descrição;
\item[d)] \textbf{Seção 5} - descrição;
\item[e)] \textbf{Seção 6} - descrição;
\item[f)] \textbf{Seção 7} - descrição;
\end{enumerate}
