\chapter{Introdução}
\label{cap:introducao}
Considerado como ponto chave para a Web Semântica \cite{berners2001semantic}, Dados Conectados vem sendo abordado de forma crescente por acadêmicos, governos e empresas. Tal crescimento se deve ao fato de Dados Conectados ser uma alternativa relevante para conectar bases de dados distribuídas na Web. Segundo \citeonline{hyland2011joy}, Dados Conectados se trata de um conjunto de boas práticas para publicar e conectar dados estruturados na Web. Ademais, \citeonline{berners2006linked} afirma que não se trata apenas de por dados na internet, mas fazer conexões entre eles, permitindo assim, que pessoas ou máquinas possam explorar a Web dos dados.

Web de Dados bem como Web dos Documentos, são termos utilizados para descrever o paradigma da Web atualmente, onde o primeiro, que tem como base o segundo, reflete a possibilidade de conectar os dados através de links. Sendo esta, uma possibilidade inexistente quando se fala de Web dos Documentos, pois nesta, os documentos apontam para outros documentos através de links.

Dados Conectados pode ser visto sob duas perspectivas: consumo e publicação. A primeira aborda o ponto de vista do consumidor, tratando a exploração de dados para tornar as aplicações mais ricas. A segunda está sob a perspectiva do publicador, abordando processos \cite{bizer2007publish, hyland2011joy, villazon2011methodological, Avila2015} e conceitos \cite{berners2006linked, wood2014linked} necessários para publicar e manter os dados na Web de forma conectada. 

Publicar ou manter dados conectados na Web vai além de disponibilizar conjuntos de dados através de serializações RDF. É necessário conectá-los a outros conjuntos de dados já existentes. Porém, criar links entre conjuntos de dados requer uma análise cuidadosa por parte do especialista, que apesar de ser uma abordagem eficaz, não é escalável, visto que a quantidade de dados publicados cresce constantemente. Consequentemente, inviabilizando o processo de publicação de forma artesanal. Logo, para que seja possível constuir a Web de Dados de forma eficiente, é necessário que existam soluções capazes de conectar dados de forma automática ou semi-automática.

Conectar dados automaticamente é um problema reconhecido por diversas comunidades. Dentre essas comunidades, podemos destacar as comunidades de Bancos de Dados e Web Semântica. Na primeira, esse problema é conhecido através do termo \textbf{\textit{Record Linkage}} \cite{gu2003record}. Na segunda, ele é reconhecido pelo termo \textbf{\textit{Instance Matching}}. Além disso, também é possível encontrar referências como \textbf{\textit{Problema de Resolução de Entidades}} \cite{menestrina2005generic} e \textbf{\textit{Deduplicação}} \cite{sarawagi2002interactive}, que se referem ao processo que tem como objetivo identificar e conectar recursos julgados representar a mesma entidade do mundo real.

Afim de conectar os recursos disponíveis na Web. A comunidade, através da \textit{Ontology Alignment Evaluation Initiative}\footnote{http://oaei.ontologymatching.org} (OAEI), realiza anualmente uma avaliação das soluções de alinhamento de dados. Tal iniciativa mostrou-se promissora, pois foi possível notar a melhora nos resultados das soluções propostas. Porém, após experimentos utilizando dados reais, \citeonline{homoceanu2014putting} concluiram que essas soluções não estão prontas para conectar dados automaticamente de forma confiável. Além disso, o baixo nível de confiabilidade dessas soluções deve-se ao fato de que os testes realizados pela OAEI não refletem os problemas apresentados por dados reais. 

Neste trabalho, nós apresentados uma abordagem semi-automática para o alinhamento de dados conectados. 

%Por exemplo, admitindo a existência de uma entidade do mundo real (e.g. Pesquisador) onde atributos como nome e endereço estão armazenados em datasets distintos, onde estão vinculados a recursos que usam as seguintes URIs (http://www.ic.ufal.br/dac/rbie/author/615 e http://lattes.cnpq.br/4038730280834132) como na Figura \ref{fig:modelo_semantico}. Neste cenário, não é possível determinar, apenas através do alinhamento entre as ontologias, que ambas URIs fazem referência a mesma entidade, sendo necessário analisar informações adicionais provenientes de conexões semânticas entre conceitos e relações presentes nestes dados [31, 35]. 

%\begin{figure}[!ht]
%	\centering
%	\includegraphics[width=0.95\textwidth]{./imagens/researcher.png}\\
%    \caption{Relação entre entidade do mundo real, dados e modelo semântico}
%	\footnotesize{Fonte: Próprio autor.}
%	\label{fig:modelo_semantico}
%\end{figure}

%AQUI, PRECISA ESCREVER SOBRE TÉCNICAS APLICADAS POR OUTROS TRABALHOS (RELACIONADOS) E DIZER POR QUE ELES NÃO RESOLVEM O PROBLEMA.  

%DEPOIS DISSO, DIZER QUE UM CAMINHO DE SOLUÇÃO É O ALINHAMENTO USANDO XXXXXXXXX E/OU YYYYYYYYYYY.

\subsection*{Contextualização}

\subsection*{Problemática}

\subsection*{Solucionática}

\subsection*{Objetivos}

\subsubsection*{Objetivo Geral}

\begin{itemize}
	\item Objetivo geral.
\end{itemize}

\subsubsection*{Objetivos específicos}
\begin{itemize}
\item Objetivo específico 1;
\item Objetivo específico 2;
\item Objetivo específico 3;
\item Objetivo específico 4.
\end{itemize}

\subsection*{Estrutura do trabalho}

O restante do trabalho está estruturado como segue:

\begin{enumerate}
\item[a)] \textbf{Seção 2} - descrição;
\item[b)] \textbf{Seção 3} - descrição;
\item[c)] \textbf{Seção 4} - descrição;
\item[d)] \textbf{Seção 5} - descrição;
\item[e)] \textbf{Seção 6} - descrição;
\item[f)] \textbf{Seção 7} - descrição;
\end{enumerate}
