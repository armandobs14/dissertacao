\chapter{Introdução}
\label{cap:introducao}
O Atual paradigma da Web é descrito como Web de documentos. Neste, os documentos ou páginas são descritas através de uma linguagem de marcação de hipertexto (HTML) e interligados através de hiperlinks. Para identificar as páginas na Web, é necessário um mecanismo capaz de identificar de forma única cada uma dessas páginas. Para isso, a identificação é realizada através de um mecanismo de identificação global chamado URI (Uniform Resource Identifier). Dessa forma, para ter acesso a um conteúdo específico de qualquer documento neste paradigma, é necessário obter todo o documento.

Diferentemente da Web de documentos, a Web de dados, através de URIs, possibilita que os dados sejam identificados de forma única, permitindo que apenas os dados desejados sejam recuperados. Os dados são representados através de um framework de descrição de recursos (RDF) além de contar como um protocolo para consulta aos dados, o SPARQL. Através desses recursos juntamente com princípios e boas práticas, Surge o conceito de Dados Conectados.

%Deve-se ressaltar que o RDF é responsável por dar estrutura aos dados seguindo o padrão de tripla: Sujeito, Predicado e Objeto. Com objetivo de dar significado aos dados utilizam-se vocabulários e ontologias. Tecnicamente, pode-se entender uma ontologia como um artefato computacional utilizado para modelar um domínio de interesse. Essa ontologia é composta de classes e propriedades, que são implementados através de uma linguagem de descrição de ontologias (OWL).
 
O conceito de Dados Conectados está intimamente relacionado à Web de dados, pois de acordo com \citeonline{bizer2009linked}, Dados Conectados trata-se de usar a Web para criar links entre dados de fontes diferentes. Além disso, Dados Conectados é considerado um ponto chave para o desenvolvimento da Web Semântica \cite{berners2001semantic} . Além disso, \citeonline{Isotani2015} destacam seu potencial de aplicação para negócios e governos.  

Na perspectiva de negócio, temos o caso da BestBuy, que través da utilização de serialização RDF (RDFa) melhorou o número de acessos via buscador  Google entre 15\% e 30\%. E do Google, que passou a utilizar a serialização json-ld em um de seus produtos, o Gmail. No governo, temos o caso do governo britânico que publica seus dados em formato RDF, permitindo que os dados possam ser conectados a outros, mesmo estando em bases diferentes. Consequentemente, a utilização de Dados Conectados vem sendo realizada de forma crescente.

Dados Conectados pode ser visto sob duas perspectivas: consumo e publicação. A primeira aborda o ponto de vista do consumidor, tratando a exploração de dados para tornar as aplicações mais ricas. A segunda está sob a perspectiva do publicador, abordando processos \cite{bizer2007publish, hyland2011joy, villazon2011methodological, Avila2015} e conceitos \cite{berners2006linked, wood2014linked} necessários para publicar e manter os dados na Web de forma conectada. 

Publicar ou manter dados conectados na Web vai além de disponibilizar conjuntos de dados através de serializações RDF. É necessário conectá-los a outros conjuntos de dados já existentes. Porém, criar links entre conjuntos de dados requer uma análise cuidadosa por parte do especialista, que apesar de ser uma abordagem eficaz, não é escalável, visto que a quantidade de dados publicados cresce constantemente. Consequentemente, inviabilizando o processo de publicação de forma artesanal. Logo, para que seja possível construir a Web de Dados de forma eficiente, é necessário que existam soluções capazes de conectar dados de forma automática ou semiautomática.

Conectar dados automaticamente é um problema reconhecido por diversas comunidades. Dentre essas comunidades, podemos destacar as comunidades de Bancos de Dados e Web Semântica. Na primeira, esse problema é conhecido através do termo \textbf{\textit{Record Linkage}} \cite{gu2003record}. Na segunda, ele é reconhecido pelo termo \textbf{\textit{Instance Matching}}. Além disso, também é possível encontrar referências como \textbf{\textit{Problema de Resolução de Entidades}} \cite{menestrina2005generic} e \textbf{\textit{Deduplicação}} \cite{sarawagi2002interactive}, que se trata de um processo que tem como objetivo identificar e conectar recursos julgados representar a mesma entidade do mundo real.

\citeonline{castano2011ontology} ressaltam que a correspondência de instâncias (Instance Matching) apresenta características adicionais tais como: (i) \textbf{heterogeneidade estrutural}, que se refere à variação na estrutura das instâncias; (ii)\textbf{ conhecimento implícito}, que se refere as características e restrições  apresentadas pelo domínio e (iii) \textbf{identificação orientada a URI}, refere-se ao reuso das URIs para identificar novas informações a respeito de instâncias já existentes. Logo, há a necessidade de soluções específicas para a execução correta do processo de correspondência de instâncias. 

De acordo com \cite{castano2011ontology}, existem diversas técnicas que podem ser utilizadas no processo de correspondência de instâncias. Algumas dessas técnicas foram adaptadas de abordagens existentes na comunidade de Banco de Dados. Como pode ser visto na Figura \ref{fig:im_techniques}, tais técnicas estão concentradas em duas categorias. A primeira se refere a abordagens orientadas a valor. Nesta categoria, assume-se que a similaridade entre recursos pode ser obtida através da correspondência entre os atributos desses recursos. Vale a pena ressaltar que a maior parte das abordagens dessa categoria estão focadas na comparação entre textos (e.g. Dice, Levenshtein, Jaro etc.).

\begin{figure}[!h]
	\centering
	\includegraphics[width=0.85\textwidth]{./imagens/im_techniques.pdf}
	\caption{Técnicas para correspondência de instâncias}
	\footnotesize{Fonte: adaptado de \cite{castano2011ontology}.}
	\label{fig:im_techniques}
\end{figure}

Na segunda categoria, estão concentradas as técnicas baseadas em recursos. 

\textbf{Baseada em aprendizagem:} utiliza grupos de trinamento bem como técnicas de aprendizagem de máquina como Máquina de vetor de suporte para definir se os recursos representam a mesma entidade do mundo real.

\textbf{Baseada em similaridade:} Essa técnica enxerga os recursos como um conjunto de valores. pode utilizar as mesmas funções para comparação entre textos, que pode utilizar a similaridade média para comparar dois recursos. 

\textbf{Baseadas em regras:} Diferente das outras técnicas, esta zero valores boleanos no lugar de valores numéricos. Além disso, esta subcategoria apresenta bons resultados, porém esta abordagem é depende do domínio. 

\textbf{Baseadas em contexto:} O contexto de um recurso pode ser entendido como os relacionamentos dele com outros recursos. Dessa forma, essa abordagem analisa as instâncias e suas relações.
Conforme as técnicas apresentadas, tem-se que o calculo da similaridade entre recursos é elemento comum entre as aplicações de correspondência de instância.

Para identificar e conectar recursos na Web, a comunidade vem apresentando um número crescente de soluções (ver Figura \ref{fig:oaei_imtools}).a \textit{Ontology Alignment Evaluation Initiative} (OAEI) realiza uma avaliação anual, que consiste em alinhar dois conjuntos de dados pré-definidos e comparar o alinhamento gerado pela solução com o alinhamento de referência. A partir da comparação entre os dois conjuntos de alinhamento, as métricas de precisão, revocação e medida-f são geradas.


\begin{figure}[!h]
        \centering
        \includegraphics[width=0.9\textwidth]{./imagens/im_tools.pdf}
    \caption{Quantidade de soluções submetidas ao OAEI}
        \footnotesize{Fonte: \cite{cheatham2015results}.}
        \label{fig:oaei_imtools}
\end{figure}

Porém, de acordo com \citeonline{homoceanu2014putting}, apesar dos bons resultados apresentados, as soluções não estão prontas para alinhar dados automaticamente de forma confiável. Para isso, foi realizado um experimento equivalente ao executado pela OAEI. Entretanto, utilizando dados reais, que foram obtidos de 5 fontes diferentes (Freebase\footnote{\url{http://www.freebase.com/}}, DBPedia\footnote{\url{http://dbpedia.org}} , LinkedMDB\footnote{\url{http://www.linkedmdb.org}}, DrugBase\footnote{\url{http://www.drugbase.de/de/}} e NewYork Times\footnote{\url{http://www.nytimes.com}}), que podem ser obtidos através do endereço \url{http://www.ifis.cs.tu-bs.de/node/2906}.

No experimento realizado por \citeonline{homoceanu2014putting} foi utilizado uma abordagem de caixa preta. Nesta abordagem, qualquer sistema independente de domínio pode ser utilizado na avaliação. Nesse contexto, o SLINT+ \cite{nguyen2012interlinking} foi utilizado por ser uma ferramenta independe de domínio e não precisa de treinamento.
A ferramenta foi avaliada em duas perspectivas. Na primeira perspectiva a transitividade da propriedade \textit{owl:sameAs} foi desconsiderada. Diferentemente da segunda, que considerou a transitividade das correspondências criadas. A tabela \ref{tab:homoceanu2014putting} apresenta os resultados obtidos no experimento.

\begin{table}[h]
	\centering
	\caption{Quantidade de links owl:sameAs, quantidade de links owl:sameAs entre tipos diferentes,precisão. Nas perspectivas sem e com transividade}
	\label{tab:homoceanu2014putting}
	\begin{tabular}{@{}l|lll|lll@{}}
		\toprule
		\multicolumn{1}{c|}{\multirow{2}{*}{$\theta$}} & \multicolumn{3}{c|}{SLINT+}                                                                      & \multicolumn{3}{c}{$cl_{TR}$}                                                                  \\
		\multicolumn{1}{c|}{}                        & \multicolumn{1}{c}{\#sameAs} & \multicolumn{1}{c}{Inter-domínio} & \multicolumn{1}{c|}{Precisão} & \multicolumn{1}{c}{\#sameAs} & \multicolumn{1}{c}{Inter-domínio} & \multicolumn{1}{c}{Precisão} \\ \midrule
		0.95                                         & 8,020                        & 33                                & 0.91                          & 2,055                        & 89                                & 0.20                         \\
		0.75                                         & 16,739                       & 119                               & 0.71                          & 5,498                        & 216                               & 0.15                         \\
		0.50                                         & 17,436                       & 230                               & 0.76                          & 7,038                        & 396                               & 0.09                         \\
		0.25                                         & 25,113                       & 1,734                             & 0.67                          & 14,879                       & 2,408                             & 0.02                         \\ \bottomrule
	\end{tabular}
\end{table}



%Além disso, o experimento permitiu notar que as características do modelo não eram considerados durante o processo de correspondência de instâncias.

Segundo \cite{homoceanu2014putting} e \cite{ferrara2008towards} uma solução de correspondência de instâncias deveria explorar a modelagem ontológica, que utiliza vocabulários e ontologias para seu desenvolvimento. Essa recomendação se deve ao fato de RDF apenas dar estrutura aos dado. Sendo responsabilidades das ontologias dar significado. Um modelo ontológico é composto por classes e propriedades, que são descritas através de uma linguagem de descrição de ontologias (e.g. OWL). As classes são utilizadas para representar os conceitos que pertencem ao domínio de interesse, já as propriedades são utilizadas para relacionar conceitos, sendo chamadas de propriedades de objeto (object properties) ou relacionar conceitos e dados, sendo chamadas de data properties. Essas propriedades possuem características como transitividade, simetria e reflexibilidade. Dessa forma, o baixo suporte às características das propriedades pode afetar diretamente na qualidade de soluções de correspondência de instâncias. 
        
Dentre as propriedades com suporte inadequado,  destaca-se a propriedade owl:sameAs, que é responsável por identificar recursos equivalentes. Além disso, essa se trata de uma propriedade transitiva, de forma que se existem dois recursos equivalentes R1 e R2 e existe um terceiro recurso R3 que é equivalente a R2, então R1 é equivalente a R3 como representado na Figura \ref{sameAsSample}. Tais características fazem com que a propriedade owl:sameAs seja uma das  mais utilizadas para alinhar dados na Web. Dessa forma, utilizar ferramentas que levem em consideração as características das propriedades é de grande importância para alinhar dados de forma confiável.


\begin{figure}[h]
	\centering
	\subfloat[owl:sameAs]
	{
		\begin{tikzpicture}[node distance=1cm, auto,]
		%nodes
		\node[ellipse,draw] (r3) {R3};
		
		\node[above=of r3] (dummy) {};
		\node[right= of dummy,ellipse,draw](r2) {R2}
		edge[pil,<->, bend left=45] node[auto] {owl:sameAs} (r3);
		
		\node[left= of dummy,ellipse,draw] (r1) {R1}
		edge[dashed,<->, bend right=45] node[auto] {owl:sameAs} (r3)
		edge[pil,<->, bend left=45] node[auto] {owl:sameAs} (r2);
		\end{tikzpicture}
	}
	\subfloat[Exemplo]
	{
		\begin{tikzpicture}[node distance=1cm, auto,]
		%nodes
		\node[ellipse,draw] (r3) {lattes:perfil\_1};
		
		\node[above=of r3] (dummy) {};
		\node[right= of dummy,ellipse,draw](r2) {dblp:perfil\_2}
		edge[pil,<->, bend left=45] node[auto] {owl:sameAs} (r3);
		
		\node[left= of dummy,ellipse,draw] (r1) {schoolar:perfil\_3}
		edge[dashed,<->, bend right] node[auto] {owl:sameAs} (r3)
		edge[pil,<->, bend left] node[auto] {owl:sameAs} (r2);
		\end{tikzpicture}
		}
		\caption{Transitividade da propriedade owl:sameAs}
		\label{sameAsSample}
\end{figure}


Neste contexto, este trabalho propõe uma abordagem independente de contexto para o alinhamento de dados conectados por meio de um processo de alinhamento que leva em consideração aspectos dos dados e  características do modelo ontológico. Assim, os recursos/instâncias analisados, além de alinhados através das propriedades de dados, podem ser alinhados través seus relacionamentos. Ademais, a proposta trata o problema do alinhamento entre \textit{datasets} reais, permitindo que seja possível alinhar \textit{datasets} distribuídos na Web de forma confiável.

\section{Objetivo}


Essa abordagem visa disponibilizar um mecanismo útil que permita alinhar semiautomaticamente recursos entre \textit{datasets} diferentes. Além disso, a proposta também pretende facilitar a identificação e alinhamento de recursos dentro do mesmo \textit{dataset}.
Dessa forma, o trabalho lida com aspectos mais gerais, como facilitar o alinhamento de dados, quanto com problemas específicos, como a definição de métricas para a análise de similaridade entre recursos, que sejam capazes de suportar os problemas provenientes dos \textit{datasets} reais. Apesar de ser um trabalho com enfoque em engenharia de software, as suas contribuições estão mais voltadas para a área de Dados Conectados. Segue algumas dessas contribuições:

\begin{itemize}
        \item Construção de um processo para alinhar dados conectados independente de contexto;
        \item Definição de uma função de similaridade que contempla problemas provenientes de \textit{datasets} reais (acentuação, ausência de propriedades, formatação e outros);
\end{itemize}

\section{Contribuição e relevância do trabalho}

Anualmente a OAEI realiza a avaliação de ferramentas para a correspondência de instâncias. Essa avaliação utiliza \textit{datasets} previamente disponibilizados juntamente com uma referência de correspondência entre as instâncias. Essa prática permite que o experimento seja reproduzido, permitindo que os resultados fornecidos pela OAEI sejam validados. Por outro lado, permite também que dos desenvolvedores entendam os \textit{datasets} e desenvolvam algoritmos dedicados à computação específica para o \textit{dataset}. Desta forma, é necessário o estudo de abordagens independentes de contexto que sejam capazes de alinhar dados de forma confiável.

Nesse contexto, a proposta apresenta as seguintes contribuições:
\begin{itemize}
\item Desenvolvimento de processo independente de contexto para o alinhamento de dados conectados;
\item Viabilização da execução do alinhamento diretamente dentro do armazenamento de triplas.
\item  Criação de experimento e estudo de caso para avaliar a eficácia das soluções de alinhamento no estado da arte.
\end{itemize}

\section{Estrutura do trabalho}

Esta dissertação está dividida em 7 capítulos. O Capítulo \ref{cap:introducao} introduz a problemática e os objetivos do trabalho proposto, enaltecendo a necessidade de uma abordagem capaz de alinhar dados conectados. No Capítulo \ref{cap:fundamentacao}, são apresentados os conceitos relacionados ao tema deste trabalho, como RDF, Ontologias, Dados Conectados, Algoritmos de similaridade e Alinhamento de Dados Conectados.

No Capítulo \ref{cap:relacionados}, são apresentados os trabalhos relacionados à abordagem proposta. Na conclusão do capítulo é apresentada uma tabela comparativa entre a abordagem proposta e os trabalhos relacionados apresentados.

O Capítulo \ref{cap:processo} descreve em detalhes o processo de correspondência de instâncias proposto. O processo foi descrito por intermédio de um diagrama de atividades. A subseção \ref{sub:cascata} descreve em detalhes como é realizado o alinhamento/correspondência em cascata, que é umas das principais contribuições deste trabalho.

O Capítulo \ref{cap:componentes} mostra como a proposta foi desenvolvida, tanto no processo de alinhamento por intermédio de um diagrama de atividades como na arquitetura através de um diagrama de componentes. Neste capítulo, todas as etapas do processo bem como os componentes da arquitetura são descritos detalhadamente.

No Capítulo \ref{cap:experimento}, um experimento foi projetado para avaliar, em termos de eficácia através das métricas de precisão, revocação e medida-f, a abordagem proposta, em comparação com AgreementMakerLite (AML) \cite{fariaoaei} e RiMOM-2016 \cite{zhang2016rimom}. Cada conjunto de alinhamento gerado é avaliado e uma discussão geral é apresentada ao final do capítulo.

No Capítulo \ref{cap:estudo} é apresentado um estudo de caso. Nele é descrita o QIE, um sistema que apresenta o cruzamento dados da Revista Brasileira de Informática na Educação (RBIE), Simpósio Brasileiro de Informática na Educação (SBIE) e Workshop de informática na Escola (WIE) com dados extraídos da plataforma LATTES. Além disso, é descrito como o processo foi utilizado para alinhar os dados dos pesquisadores e de suas produções científicas entre essas bases.


Por fim, no Capítulo \ref{cap:conclusao}, são apresentadas as considerações finais deste trabalho. bem como são definidos alguns trabalhos futuros.
